\documentclass[10pt,a4paper,english]{article}
\usepackage{babel}
\usepackage{ae}
\usepackage{aeguill}
\usepackage{shortvrb}
\usepackage[latin1]{inputenc}
\usepackage{tabularx}
\usepackage{longtable}
\setlength{\extrarowheight}{2pt}
\usepackage{amsmath}
\usepackage{graphicx}
\usepackage{color}
\usepackage{multirow}
\usepackage{ifthen}
\usepackage[DIV12]{typearea}
% generated by Docutils <http://docutils.sourceforge.net/>
\newlength{\admonitionwidth}
\setlength{\admonitionwidth}{0.9\textwidth}
\newlength{\docinfowidth}
\setlength{\docinfowidth}{0.9\textwidth}
\newlength{\locallinewidth}
\newcommand{\optionlistlabel}[1]{\bf #1 \hfill}
\newenvironment{optionlist}[1]
{\begin{list}{}
  {\setlength{\labelwidth}{#1}
   \setlength{\rightmargin}{1cm}
   \setlength{\leftmargin}{\rightmargin}
   \addtolength{\leftmargin}{\labelwidth}
   \addtolength{\leftmargin}{\labelsep}
   \renewcommand{\makelabel}{\optionlistlabel}}
}{\end{list}}
\newlength{\lineblockindentation}
\setlength{\lineblockindentation}{2.5em}
\newenvironment{lineblock}[1]
{\begin{list}{}
  {\setlength{\partopsep}{\parskip}
   \addtolength{\partopsep}{\baselineskip}
   \topsep0pt\itemsep0.15\baselineskip\parsep0pt
   \leftmargin#1}
 \raggedright}
{\end{list}}
% begin: floats for footnotes tweaking.
\setlength{\floatsep}{0.5em}
\setlength{\textfloatsep}{\fill}
\addtolength{\textfloatsep}{3em}
\renewcommand{\textfraction}{0.5}
\renewcommand{\topfraction}{0.5}
\renewcommand{\bottomfraction}{0.5}
\setcounter{totalnumber}{50}
\setcounter{topnumber}{50}
\setcounter{bottomnumber}{50}
% end floats for footnotes
% some commands, that could be overwritten in the style file.
\newcommand{\rubric}[1]{\subsection*{~\hfill {\it #1} \hfill ~}}
\newcommand{\titlereference}[1]{\textsl{#1}}
% end of "some commands"
\ifthenelse{\isundefined{\hypersetup}}{
\usepackage[colorlinks=true,linkcolor=blue,urlcolor=blue]{hyperref}
}{}
\title{Pr�sentation de RST + LaTeX}
\author{}
\date{}
\hypersetup{
pdftitle={Pr�sentation de RST + LaTeX}
}
\raggedbottom
\begin{document}
\maketitle

\setlength{\locallinewidth}{\linewidth}
% Ceci est un commentaire en RST (il faut une ligne blanche apr�s un commentaire) 
\hypertarget{table-des-mati-res}{}
\pdfbookmark[0]{Table des mati�res}{table-des-mati-res}
\subsubsection*{~\hfill Table des mati�res\hfill ~}
\label{table-des-mati-res}
\begin{list}{}{}
\item {} \href{\#les-listes-sous-section}{Les listes + sous section}
\begin{list}{}{}
\item {} \href{\#les-listes-dans-une-sous-section}{Les listes dans une sous-section}

\end{list}

\item {} \href{\#latex}{\LaTeX}
\begin{list}{}{}
\item {} \href{\#un-petit-bout-en-latex}{Un petit bout en \LaTeX}

\item {} \href{\#un-plus-gros-bout-en-latex}{Un plus gros bout en \LaTeX}

\end{list}

\end{list}



%___________________________________________________________________________

\hypertarget{les-listes-sous-section}{}
\pdfbookmark[0]{Les listes + sous section}{les-listes-sous-section}
\section*{Les listes + sous section}
\label{les-listes-sous-section}


%___________________________________________________________________________

\hypertarget{les-listes-dans-une-sous-section}{}
\pdfbookmark[1]{Les listes dans une sous-section}{les-listes-dans-une-sous-section}
\subsection*{Les listes dans une sous-section}
\label{les-listes-dans-une-sous-section}

Les listes sont un des avantages de base de RST, pas de begin{\{}itemize{\}} toutes les 5 minutes.
\begin{itemize}
\item {} \begin{description}
\item[{Item 1}] \leavevmode \newcounter{listcnt0}
\begin{list}{\arabic{listcnt0}.}
{
\usecounter{listcnt0}
\setlength{\rightmargin}{\leftmargin}
}
\item {} \begin{description}
\item[{1a}] \leavevmode \begin{itemize}
\item {} 
sous-sousitem1

\end{itemize}

\end{description}

\item {} \begin{description}
\item[{2a}] \leavevmode \begin{itemize}
\item {} 
sous-sousitem2

\end{itemize}

\end{description}

\end{list}

\end{description}

\item {} 
Item 2

\end{itemize}


%___________________________________________________________________________

\hypertarget{latex}{}
\pdfbookmark[0]{{\textbackslash}LaTeX}{latex}
\section*{\LaTeX}
\label{latex}


%___________________________________________________________________________

\hypertarget{un-petit-bout-en-latex}{}
\pdfbookmark[1]{Un petit bout en {\textbackslash}LaTeX}{un-petit-bout-en-latex}
\subsection*{Un petit bout en \LaTeX}
\label{un-petit-bout-en-latex}

Dans le titre de la sous section, il y a une insertion en ligne d'une commande latex : {\textbackslash}LaTeX, via le role d�finit plus haut : \textbf{raw-l}.


%___________________________________________________________________________

\hypertarget{un-plus-gros-bout-en-latex}{}
\pdfbookmark[1]{Un plus gros bout en {\textbackslash}LaTeX}{un-plus-gros-bout-en-latex}
\subsection*{Un plus gros bout en \LaTeX}
\label{un-plus-gros-bout-en-latex}
Soit $s$ l'absice sur la barre, l'effort tranchant :
\begin{eqnarray*}
pour \; s<l/3| \; \vec{E_t} & = & F/2\\
pour \; l/3 \leq s  < 2 l/3 | \; \vec{E_t} & =& 3 F (l/2-s)/l\\
pour \; 2 l/3 \leq  s | \vec{E_t} & = & -F/2\\
\end{eqnarray*}
\end{document}
